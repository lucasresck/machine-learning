\documentclass{article}
\usepackage[utf8]{inputenc}

\title{Problem set 1\\
    \large{Machine Learning}}
\author{Lucas Emanuel Resck Domingues\\    
    Professor: Rodrigo Targino\\\\
    {School of Applied Mathematics}\\
    {Getulio Vargas Foundation}}
\date{\today}

\usepackage{natbib}
\usepackage{graphicx}
\usepackage{amsmath}
\usepackage{amsfonts}
\usepackage{amssymb}
\usepackage{mathtools}
\usepackage{fullpage}
\usepackage{enumitem}

\begin{document}

    \maketitle

    \noindent \textbf{\textit{Exercise 1.3}}

    \noindent \textit{The weight update rule in (1.3)~\cite{yaser2012learning} has the nice interpretation that it moves in the direction of classifying $\mathbf{x}(t)$ correctly.}    
    \begin{itemize}[before=\itshape]
        \item[(a)] Show that $y(t) \mathbf{w}^\intercal(t) \mathbf{x}(t) < 0$. [Hint: $\mathbf{x}(t)$ is misclassified by $\mathbf{w}(t)$.]
        \item[(b)] Show that $y(t) \mathbf{w}^\intercal(t + 1) \mathbf{x}(t) > y(t) \mathbf{w}^\intercal(t) \mathbf{x}(t)$. [Hint: Use (1.3)~\cite{yaser2012learning}.]
        \item[(c)] As far as classifying $\mathbf{x(t)}$ is concerned, argue that the move from $\mathbf{w}(t)$ to $\mathbf{w}(t + 1)$ is a move `in the right direction'.
    \end{itemize}

    \begin{itemize}
        \item[(a)] Because $\mathbf{x}(t)$ is misclassified, $\mathbf{w}^\intercal(t) \mathbf{x}(t)$ has the opposite sign of $y(t)$.
        \item[(b)] Note that, because $x_0 = 1$, we have $\mathbf{x}(t) \ne 0$.
            So:
            \begin{align*}
                y(t) \mathbf{w}^\intercal(t + 1) \mathbf{x}(t) &= y(t) (\mathbf{w}(t) + y(t) \mathbf{x}(t))^\intercal \mathbf{x}(t) \\
                &= y(t) (\mathbf{w}^\intercal(t) + y(t) \mathbf{x}^\intercal(t)) \mathbf{x}(t) \\
                &= y(t) \mathbf{w}^\intercal(t) \mathbf{x}(t) + [y(t)]^2 \mathbf{x}^\intercal(t) \mathbf{x}(t)  \\
                &= y(t) \mathbf{w}^\intercal(t) \mathbf{x}(t) + [y(t) \lVert\mathbf{x}\lVert]^2 \\
                &> y(t) \mathbf{w}^\intercal(t) \mathbf{x}(t)
            \end{align*}
        \item[(c)] We would like to have $y(t) \mathbf{w}^\intercal(t) \mathbf{x}(t) > 0$, if the classification of $\mathbf{x}(t)$ was correct.
            So we take a greater $y(t) \mathbf{w}^\intercal(t + 1) \mathbf{x}(t)$, with the hope that it will be positive.
            If it is not the case, we can try again.
    \end{itemize}
        

    \nocite{yaser2012learning}
    \bibliographystyle{plain}
    \bibliography{references}

\end{document}