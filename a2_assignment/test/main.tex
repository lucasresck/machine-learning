\documentclass[twocolumn]{article}

% Language setting
% Replace `english' with e.g. `spanish' to change the document language
\usepackage[english]{babel}

% Set page size and margins
% Replace `letterpaper' with`a4paper' for UK/EU standard size
% \usepackage[a4paper,top=2cm,bottom=2cm,left=3cm,right=3cm,marginparwidth=1.75cm]{geometry}

% Useful packages
\usepackage{amsmath}
\usepackage{graphicx}
\usepackage[colorlinks=true, allcolors=blue]{hyperref}
\usepackage[utf8]{inputenc}
\usepackage[T1]{fontenc}
\usepackage{hyphenat}
\usepackage{fullpage}


\title{Separability of legal documents according to precedent citations}
\author{Lucas Emanuel Resck Domingues}

\begin{document}
      % https://tex.stackexchange.com/a/140679
      % https://tex.stackexchange.com/a/21126
      % https://tex.stackexchange.com/a/28110
      \twocolumn[
            \begin{@twocolumnfalse}
                  \maketitle
                  \begin{abstract}
                        ...
                  \end{abstract}
            \end{@twocolumnfalse}
            \bigskip
      ]

      \section{Introduction}

            The Brazilian Supreme Court (STF) is the highest law court in Brazil. It produces a huge number of documents during its functioning, \textit{e.g.}, there were produced more than 1 million STF decisions \cite{} between 2011 and 2020. STF is not the only institution that deals with this problem: it is spread all over the Brazilian Judicial System.

            One approach to solve this problem is the so-called precedent: when a similar case has to be decided again, this new decision can be taken based on the referenced old decision. This way, cases are solved faster. Many precedents about a subject in a court are consolidated in what is called a ``súmula'', a document that resumes the court understanding about that subject. However, the application of this understanding is not mandatory, and the judge can take a different decision. This situation can lead not only to judicial inefficiency, but also to judicial insecurity.
            
            With this situation in mind, STF was allowed, in 2004, by Constitutional Amendment, to create "Súmulas Vinculantes", which we will call here "Binding Precedents", or just BPs. They are basically the old súmulas, but with mandatory application. These BPs are frequently cited in the STF decisions.
            
            It seems trivial that documents that cite the same precedents have the same subjects, and documents that cite different precedents have different subjects, in general. However, can machine learning models and algorithms identify this pattern themselves? That is, if a trained machine learning model is presented to a document, can it predict which precedent is being cited? These questions are very relevant because, if the answer is yes, artificial intelligence algorithms can be very helpful in assisting legal experts during their analysis, considering the big number of long legal documents.
            
            This situation of predicting the correct precedent is what is being called here as ``separability'' of documents. In this assignment, this separability will consider Binding Precedents in STF decisions. In \autoref{}, ...

      \nocite{*}
      \bibliographystyle{alpha}
      \bibliography{sample}
\end{document}