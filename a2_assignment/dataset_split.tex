\subsection{Dataset split}

    Because the original dataset is very unbalanced among the Binding Precedent citations, we need to create a sample dataset that will be used in our analysis. In this sample, each document cites only one BP, to avoid confusion; documents that share the same raw texts are removed, to avoid overfitting; we only work with the five most cited BPs, for simplicity and also data limitation; the dataset is balanced among the classes, i.e., each class has the same number of documents. In the end, we work with 1401 documents of BPs 10, 37, 4, 20, and 14, totalizing 7005 documents. These are the five considered Binding Precedent texts:

    \begin{itemize}
            \item \textbf{Binding Precedent 4}: \textit{``Salvo nos casos previstos na Constituição, o salário mínimo não pode ser usado como indexador de base de cálculo de vantagem de servidor público ou de empregado, nem ser substituído por decisão judicial.''}
            \item \textbf{Binding Precedent 10}: \textit{``Viola a cláusula de reserva de plenário (CF, artigo 97) a decisão de órgão fracionário de tribunal que, embora não declare expressamente a inconstitucionalidade de lei ou ato normativo do Poder Público, afasta sua incidência, no todo ou em parte.''}
            \item \textbf{Binding Precedent 14}: \textit{``É direito do defensor, no interesse do representado, ter acesso amplo aos elementos de prova que, já documentados em procedimento investigatório realizado por órgão com competência de polícia judiciária, digam respeito ao exercício do direito de defesa.''}
            \item \textbf{Binding Precedent 20}: \textit{``A Gratificação de Desempenho de Atividade Técnico-Administrativa - GDATA, instituída pela Lei 10.404/2002, deve ser deferida aos inativos nos valores correspondentes a 37,5 (trinta e sete vírgula cinco) pontos no período de fevereiro a maio de 2002 e, nos termos do artigo 5º, parágrafo único, da Lei 10.404/2002, no período de junho de 2002 até a conclusão dos efeitos do último ciclo de avaliação a que se refere o artigo 1º da Medida Provisória 198/2004, a partir da qual passa a ser de 60 (sessenta) pontos.''}
            \item \textbf{Binding Precedent 37}: \textit{``Não cabe ao Poder Judiciário, que não tem função legislativa, aumentar vencimentos de servidores públicos sob o fundamento de isonomia.''}
    \end{itemize}

    The main goal of this work is to assert whether traditional machine learning models can ``separate'' documents according to BP citations. For supervised learning models (predicting BP based on the raw text), we will create two datasets, training and test datasets, that will be used to fit and test these supervised models, respectively. Important to say that TF-IDF and truncated SVD (\autoref{sec:document_embedding}) are only fitted in training data, to guarantee generalization. The vectors also suffer a standardization (fitted in training data), as required by some supervised models.

    Several models need us to choose their hyperparameters, e.g., support vector machine (\autoref{sec:models}). For choosing a hyperparameter of a model, we perform K-fold cross-validation in the training dataset, i.e., for each hyperparameter, training data is split into $K$ sets, and, for each set $k$, the model is fitted in all other sets and validated in set $k$; the performances are aggregated. The hyperparameter with the best performance is the winner.
