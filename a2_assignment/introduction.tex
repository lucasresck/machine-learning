\section{Introduction}

    The Brazilian Supreme Court (STF) is the highest law court in Brazil. It produces a large number of documents during its functioning, \textit{e.g.}, more than 1 million STF decisions were produced between 2011 and 2020 \cite{stf}. STF is not the only institution that deals with overload: it is spread all over the Brazilian Judicial System.

    One approach to solving this overload problem is the so-called precedent: when a similar case has to be decided again, this new decision can be taken based on the referenced old decision. This way, cases are solved faster. Many precedents about a subject in a court are consolidated in a ``súmula'', a document that resumes the court's understanding about that subject. However, the application of this understanding is not mandatory, and the judge can take a different decision. This situation can lead not only to judicial inefficiency but also to judicial insecurity: similar cases with different results.

    With this situation in mind, STF was allowed, in 2004, by Constitutional Amendment, to create ``Súmulas Vinculantes'', which we will call here ``Binding Precedents'', or just BPs. They are the old súmulas, but with mandatory application. These BPs are frequently cited in STF decisions.

    It seems trivial that documents that cite the same precedents have the same subjects, and documents that cite different precedents have different subjects, in general. However, can machine learning models and algorithms identify this pattern themselves? That is, if a trained machine learning model is presented to a document, can it predict which precedent is being cited? These questions are very relevant because, if the answer is positive, artificial intelligence algorithms can assist legal experts during their analysis, considering that legal documents many times are long and in large numbers.

    This situation of predicting the correct precedent is what is being called here as ``separability'' of documents as if the documents could be separable in high dimensional space. In this assignment, this separability will consider Binding Precedents in STF decisions. This work is organized as follows. In \autoref{sec:dataset}, we describe the BPs, our dataset, and we also make a short data exploration. In \autoref{sec:methodology}, we present the methodology that will be followed in our experiments with machine learning models, briefly introducing these models, and presenting some details that weren't covered during the course. In \autoref{sec:results}, we show the experiment results and a discussion about what was found.
